\documentclass[a4paper,12pt]{article}
\usepackage{fontspec}
\usepackage{polyglossia}
\usepackage{geometry}  % Added the missing geometry package
\usepackage{hyperref}  % Adding hyperref for proper cross-references
\usepackage{listings}  % For code listings
\usepackage{xcolor}    % For colored text/elements

\setmainlanguage{russian}

% Использование системных шрифтов macOS
\setmainfont{Times New Roman}
\setsansfont{Arial}
\setmonofont{Courier New}
\newfontfamily\cyrillicfont[Script=Cyrillic]{Times New Roman}
\newfontfamily\cyrillicfonttt[Script=Cyrillic]{Courier New} % Adding Cyrillic monospace font

\geometry{a4paper, margin=1in}

\definecolor{codegreen}{rgb}{0,0.6,0}
\definecolor{codegray}{rgb}{0.5,0.5,0.5}
\definecolor{codepurple}{rgb}{0.58,0,0.82}
\definecolor{backcolour}{rgb}{0.95,0.95,0.92}

\lstdefinestyle{csharpstyle}{
    backgroundcolor=\color{backcolour},   
    commentstyle=\color{codegreen},
    keywordstyle=\color{blue},
    numberstyle=\tiny\color{codegray},
    stringstyle=\color{codepurple},
    basicstyle=\ttfamily\footnotesize,
    breakatwhitespace=false,         
    breaklines=true,                 
    captionpos=b,                    
    keepspaces=true,                 
    numbers=left,                    
    numbersep=5pt,                  
    showspaces=false,                
    showstringspaces=false,
    showtabs=false,                  
    tabsize=2,
    language={[Sharp]C}
}

\title{Домашнее задание №2\\
Закрепление концепции Domain-Driven Design\\
и принципов проектирования Clean Architecture}
\author{Студент}
\date{Апрель 2025}

\begin{document}

\maketitle

\section{Введение}

Данный отчет описывает разработанную систему управления зоопарком с использованием принципов Domain-Driven Design (DDD) и Clean Architecture. Система разработана для автоматизации следующих бизнес-процессов: управление животными, вольерами и расписанием кормлений.

Основное внимание уделено контролю правильного расселения животных по вольерам, чтобы предотвратить размещение травоядных в вольерах с хищниками.

\section{Реализованная функциональность}

В соответствии с требованиями заказчика была реализована следующая функциональность:

\subsection{Управление животными}

\begin{enumerate}
    \item Добавление новых животных (реализовано в классе \texttt{AnimalController}, метод \texttt{Create})
    \item Удаление животных (реализовано в классе \texttt{AnimalController}, метод \texttt{Delete})
    \item Просмотр информации о животных (реализовано в классе \texttt{AnimalController}, методы \texttt{GetAll} и \texttt{GetById})
    \item Изменение статуса здоровья (реализовано в доменной модели \texttt{Animal}, методы \texttt{Treat} и \texttt{MarkAsSick})
\end{enumerate}

\subsection{Управление вольерами}

\begin{enumerate}
    \item Добавление новых вольеров (реализовано в классе \texttt{EnclosureController}, метод \texttt{Create})
    \item Удаление вольеров (реализовано в классе \texttt{EnclosureController}, метод \texttt{Delete})
    \item Просмотр информации о вольерах (реализовано в классе \texttt{EnclosureController}, методы \texttt{GetAll} и \texttt{GetById})
    \item Просмотр списка животных в вольере (реализовано в классе \texttt{EnclosureController}, метод \texttt{GetAnimalsInEnclosure})
\end{enumerate}

\subsection{Перемещение животных между вольерами}

\begin{enumerate}
    \item Перемещение животного в вольер (реализовано в классе \texttt{AnimalController}, метод \texttt{MoveToEnclosure})
    \item Удаление животного из вольера (реализовано в классе \texttt{AnimalController}, метод \texttt{RemoveFromEnclosure})
    \item Бизнес-логика проверки совместимости животных и вольеров (реализовано в классе \texttt{Enclosure}, метод \texttt{AddAnimal})
\end{enumerate}

\subsection{Управление расписанием кормлений}

\begin{enumerate}
    \item Создание новых кормлений (реализовано в классе \texttt{FeedingScheduleController}, метод \texttt{Create})
    \item Просмотр всех запланированных кормлений (реализовано в классе \texttt{FeedingScheduleController}, метод \texttt{GetAll})
    \item Просмотр кормлений по дате (реализовано в классе \texttt{FeedingScheduleController}, метод \texttt{GetByDate})
    \item Просмотр кормлений для конкретного животного (реализовано в классе \texttt{FeedingScheduleController}, метод \texttt{GetByAnimalId})
    \item Отметка о выполнении кормления (реализовано в классе \texttt{FeedingScheduleController}, метод \texttt{MarkAsCompleted})
\end{enumerate}

\subsection{Просмотр статистики зоопарка}

\begin{enumerate}
    \item Общее количество животных (реализовано в классе \texttt{ZooStatisticsController}, метод \texttt{GetAnimalCount})
    \item Статистика по видам животных (реализовано в классе \texttt{ZooStatisticsController}, метод \texttt{GetSpeciesStatistics})
    \item Статистика по состоянию здоровья животных (реализовано в классе \texttt{ZooStatisticsController}, метод \texttt{GetAnimalHealthStatistics})
    \item Информация о вольерах (реализовано в классе \texttt{ZooStatisticsController}, методы \texttt{GetEnclosureCount}, \texttt{GetEnclosureTypeStatistics}, \texttt{GetEnclosureOccupancyRate})
    \item Общее резюме зоопарка (реализовано в классе \texttt{ZooStatisticsController}, метод \texttt{GetZooSummary})
\end{enumerate}

\section{Применение Domain-Driven Design}

В рамках проекта были применены следующие концепции Domain-Driven Design:

\subsection{Использование Value Objects для примитивов}

\begin{enumerate}
    \item \texttt{FoodType} - созданный Value Object для представления типа пищи. Класс обеспечивает неизменяемость значения и корректное сравнение на равенство.
    \item \texttt{EnclosureType} - Value Object для представления типа вольера (для хищников, травоядных, птиц, аквариум). Содержит предопределенные типы и обеспечивает неизменяемость.
    \item \texttt{Gender} - перечисление, представляющее пол животного.
\end{enumerate}

Пример реализации Value Object:

\begin{lstlisting}[style=csharpstyle]
public class FoodType
{
    public string Name { get; private set; }
    
    private FoodType(string name)
    {
        if (string.IsNullOrWhiteSpace(name))
            throw new ArgumentException("Food name cannot be empty", nameof(name));
            
        Name = name;
    }
    
    public static FoodType Create(string name) => new FoodType(name);
    
    public override bool Equals(object obj)
    {
        if (obj is not FoodType other)
            return false;
            
        return string.Equals(Name, other.Name, StringComparison.OrdinalIgnoreCase);
    }
    
    public override int GetHashCode()
    {
        return Name.ToLowerInvariant().GetHashCode();
    }
}
\end{lstlisting}

\subsection{Богатая доменная модель с инкапсуляцией бизнес-правил}

\begin{enumerate}
    \item \texttt{Animal} - инкапсулирует бизнес-правила для животных, включая логику проверки корректности создания (корректные имя, вид, дата рождения) и изменения состояния (здоровье, перемещение).
    \item \texttt{Enclosure} - содержит бизнес-правила для размещения животных, проверяя совместимость хищников и травоядных, а также контроль максимальной вместимости.
    \item \texttt{FeedingSchedule} - инкапсулирует логику планирования и выполнения кормлений.
\end{enumerate}

Пример инкапсуляции бизнес-правил в доменной модели:

\begin{lstlisting}[style=csharpstyle]
public bool AddAnimal(Animal animal)
{
    if (animal == null)
        throw new ArgumentNullException(nameof(animal));
        
    if (!CanAddAnimal())
        return false;
        
    // Domain logic to determine if the animal can be placed in this enclosure
    // For example, don't put predators with herbivores
    if (Type == EnclosureType.Predator && !IsPredator(animal.Species))
        return false;
        
    if (Type == EnclosureType.Herbivore && IsPredator(animal.Species))
        return false;
        
    _animals.Add(animal);
    animal.MoveToEnclosure(Id);
    
    return true;
}
\end{lstlisting}

\subsection{Доменные события}

Реализовано два основных доменных события для обеспечения интеграции между различными частями системы:

\begin{enumerate}
    \item \texttt{AnimalMovedEvent} - возникает при перемещении животного между вольерами.
    \item \texttt{FeedingTimeEvent} - возникает при наступлении времени кормления.
\end{enumerate}

Реализованы базовые классы для работы с событиями:

\begin{enumerate}
    \item \texttt{DomainEvent} - базовый класс для всех доменных событий.
    \item \texttt{DomainEvents} - статический класс для управления событиями.
    \item \texttt{IDomainEventHandler} - интерфейс для обработчиков событий.
\end{enumerate}

\section{Реализация Clean Architecture}

Проект структурирован согласно принципам Clean Architecture:

\subsection{Domain Layer (ядро)}

Содержит основные сущности, Value Objects и бизнес-правила. Этот слой полностью независим от внешних слоев и не содержит никаких зависимостей от внешних фреймворков или библиотек.

\begin{enumerate}
    \item \texttt{Domain/Entities} - содержит основные сущности (Animal, Enclosure, FeedingSchedule)
    \item \texttt{Domain/ValueObjects} - содержит Value Objects (FoodType, EnclosureType, Gender)
    \item \texttt{Domain/Events} - содержит доменные события и их обработчики
    \item \texttt{Domain/Interfaces} - содержит интерфейсы репозиториев, используемые для взаимодействия с хранилищами данных
\end{enumerate}

\subsection{Application Layer}

Содержит бизнес-логику приложения, использует сущности из доменного слоя и определяет интерфейсы для внешних сервисов:

\begin{enumerate}
    \item \texttt{Application/Services} - содержит сервисы, реализующие бизнес-логику (AnimalTransferService, FeedingOrganizationService, ZooStatisticsService)
    \item \texttt{Application/Interfaces} - содержит интерфейсы сервисов
    \item \texttt{Application/DTOs} - содержит объекты для передачи данных между слоями
\end{enumerate}

Этот слой зависит только от Domain Layer и не имеет зависимостей от внешних фреймворков или библиотек.

\subsection{Infrastructure Layer}

Реализует интерфейсы, определенные в доменном и прикладном слоях, для взаимодействия с внешними системами и ресурсами:

\begin{enumerate}
    \item \texttt{Infrastructure/Data} - содержит реализацию хранилища данных (InMemoryContext)
    \item \texttt{Infrastructure/Repositories} - содержит реализацию репозиториев
\end{enumerate}

\subsection{Presentation Layer}

Содержит компоненты пользовательского интерфейса и API:

\begin{enumerate}
    \item \texttt{Presentation/Controllers} - содержит веб-контроллеры для обработки HTTP-запросов
\end{enumerate}

\section{Принципы Clean Architecture в проекте}

\subsection{Зависимости направлены только внутрь}

Внутренние слои не зависят от внешних:
\begin{enumerate}
    \item Domain Layer не имеет никаких зависимостей от других слоев.
    \item Application Layer зависит только от Domain Layer.
    \item Infrastructure Layer зависит от Domain и Application Layers.
    \item Presentation Layer зависит от Domain, Application, и Infrastructure Layers.
\end{enumerate}

\subsection{Зависимости между слоями через интерфейсы}

Все взаимодействия между слоями осуществляются через интерфейсы:

\begin{enumerate}
    \item В Domain Layer определены интерфейсы репозиториев (\texttt{IAnimalRepository}, \texttt{IEnclosureRepository}, \texttt{IFeedingScheduleRepository}).
    \item В Application Layer определены интерфейсы сервисов (\texttt{IAnimalService}, \texttt{IEnclosureService}, \texttt{IFeedingScheduleService}).
    \item Infrastructure Layer реализует интерфейсы из Domain Layer.
    \item Presentation Layer использует интерфейсы сервисов из Application Layer.
\end{enumerate}

\subsection{Бизнес-логика изолирована в Domain и Application слоях}

\begin{enumerate}
    \item В Domain Layer содержатся основные бизнес-правила, связанные с сущностями (проверка здоровья животных, совместимость животных и вольеров).
    \item В Application Layer содержится бизнес-логика приложения (перемещение животных между вольерами, организация кормления, сбор статистики).
    \item Infrastructure Layer отвечает только за доступ к данным.
    \item Presentation Layer отвечает только за взаимодействие с пользователем.
\end{enumerate}

\section{Заключение}

Система управления зоопарком полностью реализована с использованием принципов Domain-Driven Design и Clean Architecture. Применение этих подходов обеспечивает:

\begin{enumerate}
    \item Чистую и понятную структуру проекта
    \item Изоляцию бизнес-логики от деталей реализации инфраструктуры и представления
    \item Легкое тестирование компонентов
    \item Гибкость и расширяемость системы
\end{enumerate}

Все требуемые бизнес-процессы автоматизированы, и система позволяет эффективно управлять животными, вольерами и кормлениями, предотвращая размещение несовместимых животных в одном вольере.

\end{document}
